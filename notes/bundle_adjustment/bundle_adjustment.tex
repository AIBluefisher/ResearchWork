% This is samplepaper.tex, a sample chapter demonstrating the
% LLNCS macro package for Springer Computer Science proceedings;
% Version 2.20 of 2017/10/04
%
\documentclass[runningheads]{llncs}
%
\usepackage{graphicx}
\usepackage{epstopdf}
\usepackage{subfigure}
\usepackage{multirow}
\usepackage[english]{babel}
\usepackage[utf8]{inputenc}

\usepackage{color}
\addtolength {\topmargin}{-1truemm}
\textwidth 184truemm
\textheight 235truemm
\columnsep 4truemm
%
\evensidemargin -11truemm
\oddsidemargin -11truemm


\usepackage{subfigure} % Written by Steven Douglas Cochran
                        % This package makes it easy to put subfigures
                        % in your figures. i.e., "figure 1a and 1b"
                        % Docs are in "Using Imported Graphics in LaTeX2e"
                        % by Keith Reckdahl which also documents the graphicx
                        % package (see above). subfigure.sty is already
                        % installed on most LaTeX systems. The latest version
                        % and documentation can be obtained at:
                        % http://www.ctan.org/tex-archive/macros/latex/contrib/supported/subfigure/

\usepackage{url}       

\usepackage{amsmath}   
\interdisplaylinepenalty=2500
 
\usepackage{array}

\begin{document}
%
\title{Optimization Methods in Bundle Adjustment}
%
%\titlerunning{Abbreviated paper title}
% If the paper title is too long for the running head, you can set
% an abbreviated paper title here
%
\author{Yu Chen}
%
\authorrunning{F. Author et al.}
% First names are abbreviated in the running head.
% If there are more than two authors, 'et al.' is used.
%
\institute{Peking University, Department of Computer Science and Technology
School of Electronic Engineering and Computer Science \\
\email{1701213988@pku.edu.cn}}
%
\maketitle              % typeset the header of the contribution
%
% \begin{abstract}
% The abstract should briefly summarize the contents of the paper in
% 150--250 words.

% \keywords{First keyword  \and Second keyword \and Another keyword.}
% \end{abstract}
%
%
%
\section{Precursors}

For optimization methods with the standard format:
\begin{equation}
    \label{equ:std_opt}
	\begin{align}
	    arg\ min\ f_0(x) \\
    s.t f_i(x) \leq 0, i=1,...,m\\
    h_i(x) = 0, i=1,...,p
	\end{align}
\end{equation}

The \textbf{Lagrangian function} of (\ref{equ:std_opt}) is 
\begin{equation}
L(x,\lambda, \nu) = f_0(x) + \sum_{i=1}^m \lambda_i f_i(x) + \sum_{i=1}^p \nu_i h_i(x)
\end{equation}
$\lambda, \nu$ are called \textit{dual variables}.

The \textbf{Lagrangian dual funciton} of (\ref{equ:std_opt}) is
\begin{equation}
g(\lambda, \nu)=inf L(x,\lambda,\nu) = inf (f_0(x) + \sum_{i=1}^m \lambda_if_i(x) + \sum_{i=1}^p \nu_ih_i(x))
\end{equation}

The \textbf{conjugate function} of (\ref{equ:std_opt}) is
\begin{equation}
f^*(y)=sup (y^Tx-f(x))
\end{equation}

The dual function and conjugate function have a relationship:
\begin{equation}
g(\lambda, \nu) = inf (f_0(x) + \lambda^T(Ax-b) + \nu^T(Cx-d))
\end{equation}

%
% ---- Bibliography ----
%
% BibTeX users should specify bibliography style 'splncs04'.
% References will then be sorted and formatted in the correct style.
%
% \bibliographystyle{splncs04}
% \bibliography{mybibliography}
%
% \begin{thebibliography}{8}
% \bibitem{ref_article1}
% Author, F.: Article title. Journal \textbf{2}(5), 99--110 (2016)

% \bibitem{ref_lncs1}
% Author, F., Author, S.: Title of a proceedings paper. In: Editor,
% F., Editor, S. (eds.) CONFERENCE 2016, LNCS, vol. 9999, pp. 1--13.
% Springer, Heidelberg (2016). \doi{10.10007/1234567890}

% \bibitem{ref_book1}
% Author, F., Author, S., Author, T.: Book title. 2nd edn. Publisher,
% Location (1999)

% \bibitem{ref_proc1}
% Author, A.-B.: Contribution title. In: 9th International Proceedings
% on Proceedings, pp. 1--2. Publisher, Location (2010)

% \bibitem{ref_url1}
% LNCS Homepage, \url{http://www.springer.com/lncs}. Last accessed 4
% Oct 2017
% \end{thebibliography}
\end{document}
